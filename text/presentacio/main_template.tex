\documentclass[10pt]{beamer}
\usetheme[
%%% options passed to the outer theme
%    hidetitle,           % hide the (short) title in the sidebar
%    hideauthor,          % hide the (short) author in the sidebar
%    hideinstitute,       % hide the (short) institute in the bottom of the sidebar
%    shownavsym,          % show the navigation symbols
%    width=2cm,           % width of the sidebar (default is 2 cm)
%    hideothersubsections,% hide all subsections but the subsections in the current section
%    hideallsubsections,  % hide all subsections
%    left                % right of left position of sidebar (default is right)
  ]{Namkeen}
  
% If you want to change the colors of the various elements in the theme, edit and uncomment the following lines
% Change the bar and sidebar colors:
%\setbeamercolor{Aalborg}{fg=red!20,bg=red}
%\setbeamercolor{sidebar}{bg=red!20}
% Change the color of the structural elements:
%\setbeamercolor{structure}{fg=red}
% Change the frame title text color:
%\setbeamercolor{frametitle}{fg=blue}
% Change the normal text color background:
%\setbeamercolor{normal text}{bg=gray!10}
% ... and you can of course change a lot more - see the beamer user manual.
\usepackage[utf8]{inputenc}
\usepackage[english]{babel}
\usepackage[T1]{fontenc}
% Or whatever. Note that the encoding and the font should match. If T1
% does not look nice, try deleting the line with the fontenc.
\usepackage{helvet}
% colored hyperlinks
\newcommand{\chref}[2]{%
  \href{#1}{{\usebeamercolor[bg]{Namkeen}#2}}%
}

\title[Apprentissage Automatique Supervisé]% optional, use only with long paper titles
{Titre: Apprentissage Automatique Supervisé}

\subtitle{Sous-Titre: Machine à Vecteur de Suppor (SVM) }  % could also be a conference name

\date{\today}

\author[IVIA-AF] % optional, use only with lots of authors
{
  Elaboré par : 
  L'Association IVIA-AF\\ 
  \chref{mailto:infos@ivia-af-ai}{{\tt infos@ivia-af-ai}}
}
% - Give the names in the same order as they appear in the paper.
% - Use the \inst{?} command only if the authors have different
%   affiliation. See the beamer manual for an example

\institute[
  {\includegraphics[width=0.80\columnwidth]{images/ivia.jpg}}\\ %insert a company, department or university logo
  
] % optional - is placed in the bottom of the sidebar on every slide
{% is placed on the bottom of the title page
   Email : \texttt{infos@ivia-af-ai} \\  
  Site Web :  \texttt{http://www.ivia-af.ai/}\\
  Github : \texttt{https://github.com/IVIA-AF}
  
  
  %there must be an empty line above this line - otherwise some unwanted space is added between the university and the country (I do not know why;( )
}

% specify the logo in the top right/left of the slide
\pgfdeclareimage[height=0.7cm]{mainlogo}{images/ivia.jpg} % placed in the upper left/right corner
\logo{\pgfuseimage{mainlogo}}

% specify a logo on the titlepage (you can specify additional logos an include them in 
% institute command below
\pgfdeclareimage[height=0.8cm]{titlepagelogo}{images/ivia.jpg} % placed on the title page
%\pgfdeclareimage[height=1.5cm]{titlepagelogo2}{images/logo} % placed on the title page
\titlegraphic{% is placed on the bottom of the title page
  \pgfuseimage{titlepagelogo}
%  \hspace{1cm}\pgfuseimage{titlepagelogo2}
}

\begin{document}
% the titlepage %\wavesbg
{
\begin{frame}[plain,noframenumbering] % the plain option removes the sidebar and header from the title page
  \titlepage
\end{frame}}
%%%%%%%%%%%%%%%%

%%%%%%%%%%%%%%%%%%%%%%%%%%%%%%%%%%%%%%%%%%%%%%%
% Don't change this
{\setbeamercolor{nust}{fg=green!50,bg=yellow}
 \setbeamercolor{sidebar}{bg=green!10}
 \setbeamercolor{structure}{fg=blue}
 \setbeamercolor{frametitle}{use=structure,fg=structure.fg}
 \setbeamercolor{normal text}{bg=white!20}
 %%%%%%%%%%%%%%%%%%%%%%%%%%%%%%%%%%%%%%%%%%%%%%%
\begin{frame}{Plan}{}
\tableofcontents
\end{frame}}
%%%%%%%%%%%%%%%%

\section{Introduction} 
\subsection*{Apprentissage Supervis\'e}
%%%%%%%%%%%%%%%%%%%%%%%%%%%%%%%%%%%%%%%%%%%%%%%
% list of the themes and options
% Don't change this
{\setbeamercolor{nust}{fg=green!50,bg=yellow}
 \setbeamercolor{sidebar}{bg=green!10}
 \setbeamercolor{structure}{fg=blue}
 \setbeamercolor{frametitle}{use=structure,fg=structure.fg}
 \setbeamercolor{normal text}{bg=white!20}
 %%%%%%%%%%%%%%%%%%%%%%%%%%%%%%%%%%%%%%%%%%%%%%%
\begin{frame}{Introduction} %frame's name
%\item \textcolor{red}
\begin{itemize}
\item Consid\'erons les donn\'ees suivantes $D = \{x_{i},y_{i}\}^{n}_{i=1},$ avec $x_{i} \in \mathbb{R}^{d}$ et $y_{i} \in \{ \pm 1\} $. 
\item L'objectif en \textcolor{red}{apprentissage supervis\'e} est de reconstruire une certaine fonction  inconnue \textcolor{red}{f} de telle sorte que: 
\textcolor{green!10!blue}{	
\begin{equation}
y \approx f(x), \quad x\in \mathbb{R}^{n\times d}.
\end{equation}}
\pause
 \item En \textcolor{red}{apprentissage supervis\'e}, nous avons des algorithmes de \textcolor{red}{classification} et de \textcolor{red}{r\'egression}
 \item Comme algorithmes de classification, nous avons la r\'egression logistique, KNN, \textcolor{blue}{Machine \`a vecteurs de Support}
 \end{itemize}
 \end{frame}}
{\setbeamercolor{nust}{fg=green!50,bg=yellow}
 \setbeamercolor{sidebar}{bg=green!10}
 \setbeamercolor{structure}{fg=blue}
 \setbeamercolor{frametitle}{use=structure,fg=structure.fg}
 \setbeamercolor{normal text}{bg=white!20}
 %%%%%%%%%%%%%%%%%%%%%%%%%%%%%%%%%%%%%%%%%%%%%%%
\begin{frame}{Introduction}{Suite}
 \begin{itemize}
\item Use item
\end{itemize}
\end{frame}}
%%%%%%%%%%%%%%%%
\section{Machine \`a vecteurs de Support (SVM)}
\subsection*{Probl\`eme}

%%%%%%%%%%%%%%%%%%%%%%%%%%%%%%%%%%%%%%%%%%%%%%%
% list of the themes and options
{\setbeamercolor{nust}{fg=green!50,bg=yellow}
 \setbeamercolor{sidebar}{bg=green!10}
 \setbeamercolor{structure}{fg=blue}
 \setbeamercolor{frametitle}{use=structure,fg=structure.fg}
 \setbeamercolor{normal text}{bg=white!20}
 %%%%%%%%%%%%%%%%%%%%%%%%%%%%%%%%%%%%%%%%%%%%%%%
\begin{frame}{Problème}
  Consid\'erons la figure ....
\begin{itemize}
	\item On constate de la figure (ajouter ref)  qu'on peut s\'eparer nos donn\'ees en \textcolor{red}{deux classes}.
	\item Pour s\'eparer ces donn\'ees, il y'a une multitude de droites que nous pouvons utiliser. 
\end{itemize}
\end{frame}}

%%%%%%%%%%%%%%%%%%%%%%%%%%%%%%%%%%%%%%%%%%%%%%%
% list of the themes and options
{\setbeamercolor{nust}{fg=green!50,bg=yellow}
 \setbeamercolor{sidebar}{bg=green!10}
 \setbeamercolor{structure}{fg=blue}
 \setbeamercolor{frametitle}{use=structure,fg=structure.fg}
 \setbeamercolor{normal text}{bg=white!20}
 %%%%%%%%%%%%%%%%%%%%%%%%%%%%%%%%%%%%%%%%%%%%%%%
\section{Calcul de la Marge}
\begin{frame}{Marge}
  \begin{block}{La formulation}
  \begin{itemize}
     \item \textcolor{red}{Marge}: Elle est donn\'ee par:
     \textcolor{green!10!blue}{	
		\begin{equation}
		2\frac{ \mid \omega^{T}x_{i}+b \mid}{\mid\mid w \mid\mid}, \quad \text{avec}\quad \mid \omega^{T}x_{i}+b \mid=1.
		\end{equation}}
	\item Donc la \textcolor{red}{marge} \'equivaut \`a:
	\textcolor{green!10!blue}{	
		\begin{equation}
		\frac{2}{\mid\mid w \mid\mid}, \quad \text{avec}\quad \mid \omega^{T}x_{i}+b \mid=1.
		\end{equation}}
  \end{itemize}
  \end{block}
\end{frame}}

%%%%%%%%%%%%%%%%%%%%%%%%%%%%%%%%%%%%%%%%%%%%%%%
% list of the themes and options
{\setbeamercolor{nust}{fg=green!50,bg=yellow}
 \setbeamercolor{sidebar}{bg=green!10}
 \setbeamercolor{structure}{fg=blue}
 \setbeamercolor{frametitle}{use=structure,fg=structure.fg}
 \setbeamercolor{normal text}{bg=white!20}
 %%%%%%%%%%%%%%%%%%%%%%%%%%%%%%%%%%%%%%%%%%%%%%%
\begin{frame}{Marge }{}
  \begin{block}{Exemple}
    \begin{enumerate}
      \item To do 
      \item Add 
      \item You can use this {\tt <dirstruct>} in your test\\
    {\tt <somewhere>\textbackslash texmf}\\
    \end{enumerate}
  \end{block}
\end{frame}}
%%%%%%%%%%%%%%%%
\section{Feedback}
\subsection{Contact Information}
% contact information
{\setbeamercolor{nust}{fg=green!50,bg=yellow}
 \setbeamercolor{sidebar}{bg=green!10}
 \setbeamercolor{structure}{fg=blue}
 \setbeamercolor{frametitle}{use=structure,fg=structure.fg}
 \setbeamercolor{normal text}{bg=white!20}
\begin{frame}{Feedback}{Contact Information}
Si vous avez de commentaires ou suggestions sur notre mod\`ele de pr\'esentation, n'h\'esitez pas \`a nous contacter. Vous pouvez voir nos contacts ci-dessous.
  \begin{center}
    \insertauthor\\
    \chref{https://github.com/IVIA-AF}{https://github.com/IVIA-AF}\\
    AMMI\\
    African Institute for Mathematical Sciences
  \end{center}
\end{frame}}
%%%%%%%%%%%%%%%%

{%\wavesbg%
\begin{frame}[plain,noframenumbering]%
  \finalpage{Merci pour avoir utilis\'e ce mod\`ele}
\end{frame}}
%%%%%%%%%%%%%%%%
\end{document}
